\newcommand{\N}{\mathbb{N}}
\newcommand{\R}{\mathbb{R}}
\newcommand{\x}{\times}						
\newcommand{\id}{\mathrm{id}}
\newcommand{\Bcal}{\mathcal{B}}
\newcommand{\Ecal}{\mathcal{E}}
\newcommand{\Ucal}{\mathcal{U}}
\newcommand{\Vcal}{\mathcal{V}}
\newcommand{\Wcal}{\mathcal{W}}
\newcommand{\Fcal}{\mathcal{F}}
\newcommand{\Ccal}{\mathcal{C}}
\newcommand{\Xcal}{\mathcal{X}}
\newcommand{\Ycal}{\mathcal{Y}}
\newcommand{\Zcal}{\mathcal{Z}}
\newcommand{\Scal}{\mathcal{S}}
\newcommand{\Lcal}{\mathcal{L}}
\newcommand{\Ical}{\mathcal{I}}
\newcommand{\Mcal}{\mathcal{M}}
\newcommand{\Ncal}{\mathcal{N}}
\newcommand{\Pcal}{\mathcal{P}}
\newcommand{\fa}{\mathfrak{a}}


\newcommand{\sm}{\setminus}						
\newcommand{\ins}{\subseteq} 					
\newcommand{\sni}{\supseteq} 					

%or mathrel? instead of mathbin
\newcommand*{\tut}[1][]{\mathrel{\tikz [baseline=-0.25ex,-, #1] \draw [#1] (0pt,0.5ex) -- (1.3em,0.5ex);}}
\newcommand*{\tnt}[1][]{\mathrel{\tikz [baseline=-0.25ex,-, #1] \draw [#1] (0pt,0.5ex) -- node[strike out,draw,-]{} (1.3em,0.5ex);}}
\newcommand*{\tuh}[1][]{\mathrel{\tikz [baseline=-0.25ex,-latex, #1] \draw [#1] (0pt,0.5ex) -- (1.3em,0.5ex);}}
\newcommand*{\hut}[1][]{\mathrel{\tikz [baseline=-0.25ex,latex-, #1] \draw [#1] (0pt,0.5ex) -- (1.3em,0.5ex);}}
\newcommand*{\huh}[1][]{\mathrel{\tikz [baseline=-0.25ex,latex-latex, #1] \draw [#1] (0pt,0.5ex) -- (1.3em,0.5ex);}}
\newcommand*{\ouo}[1][]{\mathrel{\tikz [baseline=-0.25ex,-, #1] \draw [
decoration={markings, mark=at position 0 with {\draw circle (1pt);}, 
mark=at position 1 with {\draw circle (1pt);}}, postaction={decorate},#1] (0pt,0.5ex) -- (1.3em,0.5ex);}}
\newcommand*{\tuo}[1][]{\mathrel{\tikz [baseline=-0.25ex,-, #1] \draw [
decoration={markings, mark=at position 1 with {\draw circle (1pt);}}, postaction={decorate},#1] (0pt,0.5ex) -- (1.3em,0.5ex);}}
\newcommand*{\huo}[1][]{\mathrel{\tikz [baseline=-0.25ex,latex-, #1] \draw [
decoration={markings, mark=at position 1 with {\draw circle (1pt);}}, postaction={decorate},#1] (0pt,0.5ex) -- (1.3em,0.5ex);}}
\newcommand*{\out}[1][]{\mathrel{\tikz [baseline=-0.25ex,-, #1] \draw [
decoration={markings, mark=at position 0 with {\draw circle (1pt);}}, postaction={decorate},#1] (0pt,0.5ex) -- (1.3em,0.5ex);}}
\newcommand*{\ouh}[1][]{\mathrel{\tikz [baseline=-0.25ex,-latex, #1] \draw [
decoration={markings, mark=at position 0 with {\draw circle (1pt);}}, postaction={decorate},#1] (0pt,0.5ex) -- (1.3em,0.5ex);}}
%strike out
\newcommand*{\toh}[1][]{\mathrel{\tikz [baseline=-0.25ex,-latex, #1] \draw [
decoration={markings, mark=at position 0.4 with {\draw[fill] circle (0.8pt);}},
        postaction={decorate},#1] (0pt,0.5ex) -- (1.3em,0.5ex);}}
\newcommand*{\hot}[1][]{\mathrel{\tikz [baseline=-0.25ex,latex-, #1] \draw [
decoration={markings, mark=at position 0.6 with {\draw[fill] circle (0.8pt);}},
        postaction={decorate},#1] (0pt,0.5ex) -- (1.3em,0.5ex);}}
\newcommand*{\tot}[1][]{\mathrel{\tikz [baseline=-0.25ex,-, #1] \draw [
decoration={markings, mark=at position 0.5 with {\draw[fill] circle (0.8pt);}},
        postaction={decorate},#1] (0pt,0.5ex) -- (1.3em,0.5ex);}}				
\newcommand*{\hoh}[1][]{\mathrel{\tikz [baseline=-0.25ex,latex-latex, #1] \draw [
decoration={markings, mark=at position 0.5 with {\draw[fill] circle (0.8pt);}},
        postaction={decorate},#1] (0pt,0.5ex) -- (1.3em,0.5ex);}}

\newcommand*{\rars}{\begin{array}{c}\huh\\[-10pt]\tuh\end{array}}
\newcommand*{\lars}{\begin{array}{c}\huh\\[-10pt]\hut\end{array}}
\newcommand*{\allars}{\begin{array}{c}\huh\\[-10pt]\tuh\\[-10pt]\hut\\[-10pt]\tut\end{array}}
\newcommand*{\dars}{\begin{array}{c}\huh\\[-10pt]\tuh\\[-10pt]\hut\end{array}}

%\newcommand{\tuh}{\rightarrow}
%\newcommand{\hut}{\leftarrow}
%\newcommand{\huh}{\leftrightarrow}
%\newcommand{\tut}{-}


\newcommand{\ot}{\leftarrow}
\newcommand{\oto}{\leftrightarrow}



\newcommand{\srj}{\twoheadrightarrow}
\newcommand{\inj}{\hookrightarrow}

\renewcommand{\Pr}{\mathbb{P}} 		
\newcommand{\E}{\mathbb{E}}
% \newcommand{\I}{\mathbbm{1}}
% \newcommand{\I}{\mathds{1}}
\newcommand{\Pa}{\mathrm{Pa}} 		
\newcommand{\pa}{\mathrm{pa}} 		
\newcommand{\Ch}{\mathrm{Ch}} 		
\newcommand{\Anc}{\mathrm{Anc}} 		
\newcommand{\AnCl}{\mathrm{AnCl}} 
\newcommand{\Desc}{\mathrm{Desc}} 	
\newcommand{\Dist}{\mathrm{Dist}} 
\newcommand{\Pred}{\mathrm{Pred}} 
\newcommand{\Sc}{\mathrm{Sc}}
\newcommand{\NonDesc}{\mathrm{NonDesc}}
%\DeclareMathOperator*{\Indep}{\perp\!\!\!\perp} 
%\DeclareMathOperator*{\nIndep}{\cancel\Indep} 
\DeclareMathOperator*{\Indep}{{\,\perp\mkern-12mu\perp\,}}
\DeclareMathOperator*{\nIndep}{{\,\not\mkern-1mu\perp\mkern-12mu\perp\,}}
\DeclareMathOperator*{\given}{|}
\DeclareMathOperator*{\diag}{\mathrm{diag}}
\DeclareMathOperator{\doit}{do}
\newcommand{\moral}{\mathrm{mor}}	
\newcommand{\marg}{\mathrm{mar}}
\newcommand{\aug}{\mathrm{aug}}		
\newcommand{\can}{\mathrm{can}}	
\newcommand{\ass}{\mathrm{ass}}		
\newcommand{\augacy}{\mathrm{augacy}}
\newcommand{\acag}{\mathrm{acag}}			
\newcommand{\acy}{\mathrm{acy}}	
\newcommand{\ske}{\mathrm{ske}}	
\newcommand{\HEDG}{HEDG}

\newcommand{\ReLU}{\mathrm{ReLU}} 

\newcommand{\lp}{\left ( }
\newcommand{\rp}{\right ) }
\newcommand{\lB}{\left [ }
\newcommand{\rB}{\right ] }
\newcommand{\lC}{\left \{ }
\newcommand{\rC}{\left \} }

\newcommand{\hyle}[2]{\emph{#2}}
\newcommand{\hyte}[2]{\emph{#2}}
\newcommand{\hyl}[2]{\hyperlink{#1}{#2}}
\newcommand{\hyt}[2]{\hypertarget{#1}{#2}}

\DeclareMathOperator*{\argmin}{arg\,min}
\DeclareMathOperator*{\argmax}{arg\,max}
\newcommand{\xto}[1]{\stackrel{#1}{\to}}
\newcommand{\eRN}{\overline{\mathbb{R}}}
\newcommand{\RN}{\mathbb{R}}
